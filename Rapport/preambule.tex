\documentclass{rapportCS}

%------------ Debut Packages ----------------
\usepackage[T1]{fontenc}
\usepackage[utf8]{inputenc}
\usepackage{lmodern}
\usepackage[french]{babel}
\usepackage{xcolor} % pour ajouter de la couleur (si besoin)
\usepackage{multirow}
\usepackage{makecell}
\usepackage{float}
\usepackage{tablefootnote}
\usepackage{hyperref}
\usepackage{bookmark}
\usepackage{pdfpages}
\usepackage[linesnumbered,ruled,vlined]{algorithm2e}
\usepackage{lipsum}
\usepackage{listings}
\usepackage{subcaption}
\usepackage{lscape}
\usepackage{url}
\usepackage{natbib}
\usepackage{enumitem} % Package pour la personnalisation des listes
\usepackage{pgfgantt} % Package pour les diagrammes de Gantt

% Configuration de natbib
\setcitestyle{numbers,square}

% Pour personnaliser les entêtes et pieds de pages
\usepackage{fancyhdr}

% On définit le style pour les pages "normales"
\pagestyle{fancy}
\renewcommand{\chaptermark}[1]{\markboth{\chaptername \ \thechapter.\ #1}{}} % sert à personnaliser l'affichage de \leftmark (ici : le mot "Chapitre", le numéro, un point, et le titre du chapitre, sans écrire en majuscules)
\renewcommand{\sectionmark}[1]{\markright{\thesection.\ #1}} % sert à personnaliser l'affichage de \rightmark (ici : le numéro et le titre de la section en cours, sans écrire en majuscules)
\renewcommand{\thesection}{\arabic{section}}

%------------ Fin Packages ----------------