\chapter{Étude Bibliographique}

\section{Introduction}
L’alimentation occupe une place centrale dans la santé, le bien-être et l’organisation du quotidien. Pourtant, la gestion des repas et des achats reste souvent guidée par des habitudes et des contraintes pratiques (temps, budget, disponibilité des produits), ce qui peut conduire à des choix peu adaptés aux objectifs nutritionnels et à une part importante de pertes et de gaspillage. Dans ce contexte, les applications numériques jouent un rôle croissant : elles facilitent le suivi, l’accès à l’information et la personnalisation des recommandations, tout en cherchant à s’intégrer de manière fluide dans la routine de l’utilisateur.

L’objectif de cette étude bibliographique est de situer le projet Hunger-Talk par rapport aux travaux existants sur quatre axes complémentaires : la réduction du gaspillage et la gestion de stock au niveau des ménages, le suivi nutritionnel via les applications mobiles, la recommandation de recettes et de produits alimentaires, et enfin l’usage d’agents conversationnels et de modèles de langage pour fournir une assistance personnalisée. Nous mettons également l’accent sur les approches récentes de génération augmentée par récupération (RAG) et sur la recherche vectorielle, car elles constituent une base technique pertinente pour l’assistant IA intégré à Hunger-Talk.

\section{Gaspillage alimentaire et gestion de stock au niveau des ménages}
\subsection{Enjeux et ampleur du phénomène}
Les rapports internationaux soulignent que les pertes et le gaspillage alimentaire représentent un enjeu à la fois économique, social et environnemental. Les travaux de la FAO mettent en évidence l’ampleur des pertes sur l’ensemble de la chaîne, et rappellent que la réduction du gaspillage constitue un levier important pour améliorer la durabilité des systèmes alimentaires \cite{fao2011foodlosseswaste}. Dans la même logique, le rapport \emph{Food Waste Index} des Nations Unies insiste sur la contribution du gaspillage au niveau des ménages et sur la nécessité d’outils de mesure et d’actions concrètes pour progresser \cite{unep2021foodwasteindex}.

\subsection{Pratiques domestiques et leviers d’action}
Au niveau des ménages, le gaspillage est rarement lié à un seul facteur. Il résulte plutôt d’un ensemble de pratiques (planification, stockage, préparation, gestion des restes) et de contraintes quotidiennes. La revue systématique de Schanes \emph{et al.} montre que les comportements sont influencés par des dimensions sociales et organisationnelles, et que les politiques publiques et les interventions doivent tenir compte de ces réalités de terrain \cite{schanes2018foodwastematters}. Dans un projet comme Hunger-Talk, cela se traduit par un besoin de fonctionnalités qui rapprochent l’information de l’action : visualisation du stock, anticipation des dates de péremption, et recommandations de recettes adaptées à ce qui est réellement disponible.

\section{Suivi nutritionnel et applications mobiles (mHealth)}
\subsection{Auto-surveillance et efficacité}
Le suivi alimentaire et nutritionnel est souvent associé à l’auto-régulation : mesurer ce que l’on consomme aide à prendre conscience des apports et à ajuster ses choix. Une revue systématique de la littérature sur l’auto-surveillance dans la perte de poids confirme l’intérêt du suivi, tout en soulignant que l’adhésion et la charge de saisie constituent des limites récurrentes \cite{burke2011selfmonitoringreview}. Autrement dit, les bénéfices potentiels existent, mais ils dépendent fortement de la capacité de l’outil à rester utilisable sur la durée.

\subsection{Vers des saisies plus naturelles et moins coûteuses}
Les travaux récents explorent des modes de saisie plus flexibles (multimodalité, questions de clarification, récupération d’informations) afin de réduire l’effort demandé à l’utilisateur et d’améliorer la qualité des données. L’étude \emph{SnappyMeal} illustre cette direction en proposant une application de \emph{food logging} basée sur des entrées multimodales et une récupération d’informations complémentaires, avec une évaluation \emph{in-the-wild} sur plusieurs semaines \cite{bakar2025snappymeal}. Ces approches sont particulièrement pertinentes pour Hunger-Talk, car elles rejoignent l’idée d’un suivi intégré au quotidien plutôt qu’un suivi perçu comme une tâche administrative.

\section{Systèmes de recommandation appliqués à l’alimentation}
\subsection{Fondements des systèmes de recommandation}
Les systèmes de recommandation visent à proposer des éléments pertinents à un utilisateur à partir d’informations explicites (préférences, évaluations) et implicites (historique, contexte). Le panorama classique d’Adomavicius et Tuzhilin présente les principales familles d’approches (filtrage collaboratif, contenu, hybrides) ainsi que les défis de personnalisation, de scalabilité et d’évaluation \cite{adomavicius2005recommendersurvey}. Ces concepts forment une base solide pour aborder la recommandation de recettes, où la pertinence ne se limite pas à “ce qui plaît”, mais doit intégrer des contraintes (ingrédients disponibles, objectifs nutritionnels, allergies, temps de préparation).

\subsection{Spécificités du domaine alimentaire}
Le domaine alimentaire possède des particularités qui complexifient la recommandation : la forte dépendance au contexte, la diversité culturelle, la temporalité (saisonnalité, planification), et la nécessité d’expliquer les choix de manière crédible. Trattner et Elsweiler synthétisent les contributions et défis des \emph{food recommender systems}, en montrant que les systèmes doivent souvent combiner plusieurs signaux (préférences, santé, disponibilité) et proposer des stratégies adaptées (recommandation de recettes, de menus, de produits) \cite{trattner2017foodrecsys}. Dans Hunger-Talk, la gestion de stock ajoute une contrainte forte mais utile : une recommandation “réaliste” est celle qui valorise d’abord ce que l’utilisateur possède déjà.

\subsection{Données et apprentissage autour des recettes}
La qualité d’un système de recommandation dépend aussi des données disponibles. Dans la littérature, des corpus structurés de recettes et d’images ont été proposés pour apprendre des représentations utiles aux tâches de recherche et de recommandation. Le dataset Recipe1M+ constitue une référence importante dans ce cadre, en fournissant un volume conséquent de recettes et d’images et en permettant l’apprentissage d’embeddings multimodaux \cite{marin2021recipe1mplus}. Même si Hunger-Talk ne vise pas nécessairement l’analyse d’images dans sa version actuelle, ces travaux montrent l’intérêt des représentations distribuées (embeddings) pour rapprocher les contenus (recettes) et les contraintes (ingrédients, thèmes, styles culinaires).

\section{Agents conversationnels et interaction en langage naturel}
\subsection{État de l’art des agents conversationnels en santé}
Les agents conversationnels (chatbots) sont étudiés depuis plusieurs années en santé numérique, notamment pour l’accompagnement, la prévention et l’éducation thérapeutique. La revue systématique de Laranjo \emph{et al.} met en évidence un champ en développement, avec des évaluations encore hétérogènes et des besoins de protocoles plus robustes pour mesurer efficacité, sécurité et acceptabilité \cite{laranjo2018conversationalagents}. Cette littérature rappelle qu’un agent conversationnel utile n’est pas seulement “capable de parler”, mais doit s’inscrire dans un objectif clair, avec des réponses cohérentes et une expérience utilisateur maîtrisée.

\subsection{Chatbots et recommandation nutritionnelle}
Plus récemment, l’arrivée des modèles de langage a relancé l’intérêt pour des assistants plus interactifs et explicatifs. Le cadre \emph{ChatDiet} propose un exemple de chatbot orienté recommandation nutritionnelle qui combine modèles de population, personnalisation et orchestration vers un LLM, afin d’améliorer l’explicabilité et l’adaptation à l’utilisateur \cite{yang2024chatdiet}. Pour Hunger-Talk, ces travaux renforcent l’idée qu’un assistant conversationnel peut être pertinent à condition d’être connecté à des données contextuelles (stock, objectifs, préférences) et de cadrer strictement la génération.

\section{Modèles de langage, Transformers et génération augmentée par récupération (RAG)}
\subsection{Transformers et modèles de fondation}
Les progrès récents en traitement automatique du langage reposent largement sur l’architecture Transformer, introduite par Vaswani \emph{et al.}, qui a permis de mieux capturer les dépendances contextuelles et d’entraîner des modèles à grande échelle \cite{vaswani2017attention}. Sur cette base, des modèles de fondation ont été développés avec des performances élevées sur un large éventail de tâches. La famille LLaMA a contribué à démocratiser l’accès à des modèles performants entraînés sur de grands corpus \cite{touvron2023llama}, tandis que les versions plus récentes comme Llama~3 renforcent les capacités de génération et de suivi d’instructions \cite{grattafiori2024llama3}. Dans un assistant alimentaire, ces modèles apportent un gain évident en qualité de dialogue, mais ils doivent être encadrés pour limiter les erreurs factuelles et les réponses trop générales.

\subsection{Principe du RAG et intérêt pour la fiabilité}
La génération augmentée par récupération (RAG) répond à une limite fréquente des LLM : le modèle peut produire une réponse fluide sans garantir la véracité ni la traçabilité. Le principe du RAG est de compléter la génération par une récupération de documents pertinents dans une mémoire externe (index), afin d’ancrer la réponse sur des éléments observables \cite{lewis2020rag}. Cette approche est particulièrement adaptée à Hunger-Talk, car la connaissance “utile” est en grande partie locale et dynamique : stock de l’utilisateur, préférences, contraintes, historiques, et informations nutritionnelles issues de sources structurées.

\subsection{Recherche dense, embeddings et indexation vectorielle}
Pour réaliser une récupération efficace, une tendance majeure consiste à utiliser des représentations denses (embeddings) et un schéma de recherche basé sur la similarité. Dense Passage Retrieval (DPR) illustre cette approche avec un bi-encodeur qui apprend des embeddings pour les requêtes et les passages, améliorant la récupération par rapport à des méthodes lexicales sur certains scénarios \cite{karpukhin2020dpr}. À l’échelle, l’indexation et la recherche approximative deviennent cruciales ; le travail de Johnson \emph{et al.} sur la recherche de similarité à grande échelle avec GPU (FAISS) montre comment optimiser ces opérations pour des volumes importants \cite{johnson2017faiss}. Dans une architecture RAG, ces éléments conditionnent directement la qualité finale : une bonne génération dépend d’abord d’un contexte récupéré pertinent.

\section{Architecture logicielle et déploiement}
\subsection{APIs et principes REST}
Du point de vue logiciel, les applications modernes s’appuient souvent sur des APIs pour séparer clairement la partie mobile, la logique métier et la couche données. Les principes REST proposés par Fielding fournissent un cadre de conception pour des services web évolutifs, interopérables et facilement maintenables \cite{fielding2000rest}. Dans Hunger-Talk, cette séparation facilite l’évolution : ajout de nouvelles fonctionnalités côté mobile, adaptation des endpoints, et intégration progressive des services IA.

\subsection{Reproductibilité et conteneurisation}
Le déploiement et la reproductibilité deviennent des enjeux importants dès qu’une application combine plusieurs composants (API, base de données, service IA). Docker est souvent mobilisé pour standardiser l’environnement d’exécution et faciliter la portabilité. Boettiger propose une introduction orientée reproductibilité, en montrant comment la conteneurisation peut réduire les écarts entre environnements et améliorer la réutilisabilité des configurations \cite{boettiger2015docker}. Cette perspective est utile pour un projet comme Hunger-Talk, où l’objectif est de maîtriser l’ensemble de la chaîne, y compris l’exécution locale d’un modèle de langage.

\subsection{Technologies utilisées (documentation officielle)}
Sur le plan de l’implémentation, Hunger-Talk s’appuie sur un backend Python exposé via FastAPI, une application mobile Flutter, une base de données PostgreSQL, et un moteur d’inférence locale via Ollama. Pour ces choix, la documentation officielle constitue la source de référence pour les API, les bonnes pratiques de configuration et l’usage des composants \cite{fastapidocs2025,flutterdocs2025,postgresqldocs2025,ollamadocs2025}. Ces technologies ne remplacent pas la littérature scientifique, mais elles sont indispensables pour transformer les concepts (recommandation, RAG, suivi) en une solution opérationnelle et maintenable.

\section{Synthèse et positionnement de Hunger-Talk}
Cette étude bibliographique met en évidence une convergence entre des besoins concrets (réduire le gaspillage, mieux organiser les repas, suivre ses objectifs) et des approches techniques matures (recommandation, agents conversationnels, récupération vectorielle). Les rapports sur le gaspillage alimentaire \cite{fao2011foodlosseswaste,unep2021foodwasteindex} et les revues sur les pratiques domestiques \cite{schanes2018foodwastematters} confirment la pertinence d’un outil centré sur la gestion du stock et l’accompagnement. Les travaux sur le suivi nutritionnel \cite{burke2011selfmonitoringreview} montrent l’intérêt de l’auto-surveillance, à condition de réduire la friction d’usage, ce que cherchent précisément les approches multimodales \cite{bakar2025snappymeal}.

Par ailleurs, la recommandation appliquée à l’alimentation doit être comprise comme un problème sous contraintes, où la personnalisation se combine avec des impératifs de faisabilité (ingrédients) et de santé \cite{adomavicius2005recommendersurvey,trattner2017foodrecsys}. Enfin, l’intégration d’un assistant conversationnel basé sur un LLM nécessite un cadre de fiabilisation : l’architecture RAG \cite{lewis2020rag}, combinée à la récupération dense \cite{karpukhin2020dpr} et à l’indexation vectorielle \cite{johnson2017faiss}, offre une stratégie cohérente pour produire des réponses plus ancrées dans le contexte utilisateur. C’est précisément ce positionnement — application mobile centrée stock/nutrition, recommandation contextualisée, et assistant conversationnel “augmenté” — qui guide la conception de Hunger-Talk.


