\chapter*{Introduction Générale}
\label{chapter:introduction}
\addcontentsline{toc}{chapter}{Introduction Générale}


\textbf{Contexte et Importance du Projet}

Dans un contexte où les préoccupations liées à la nutrition, à la santé et à la réduction du gaspillage alimentaire prennent une importance croissante, le développement d'applications intelligentes dédiées à la gestion alimentaire représente un enjeu majeur. Les ménages font face à de nombreux défis dans leur quotidien alimentaire : la gestion du stock de produits, la planification des repas, le suivi des objectifs nutritionnels et la réduction du gaspillage alimentaire. C'est dans cette optique que s'inscrit ce projet de fin d'année : le développement d'une application mobile intelligente, Hunger-Talk, qui combine gestion de stock, recommandations culinaires, suivi nutritionnel et interaction avec l'intelligence artificielle pour accompagner les utilisateurs vers une alimentation plus saine et mieux organisée.

\textbf{Problématique et Motivation}

Les défis liés à la gestion alimentaire domestique sont multiples et complexes. D'une part, le gaspillage alimentaire représente un problème majeur, avec des pertes significatives au niveau des ménages dues à une mauvaise gestion du stock et à l'oubli de produits proches de leur date de péremption. D'autre part, la planification des repas et la recherche de recettes adaptées aux produits disponibles restent souvent fastidieuses et peu optimisées. Enfin, le suivi des objectifs nutritionnels nécessite une attention constante et des outils adaptés pour être efficace.

Les technologies modernes, notamment l'intelligence artificielle et les architectures de type RAG (Retrieval Augmented Generation), offrent aujourd'hui des opportunités prometteuses pour résoudre ces problématiques. En combinant la gestion intelligente de données avec des modèles de langage avancés, il devient possible de créer des assistants conversationnels capables de comprendre le contexte de l'utilisateur, d'analyser son stock disponible et de proposer des recommandations personnalisées et pertinentes. Cette approche permet non seulement d'améliorer l'expérience utilisateur, mais aussi de contribuer à la réduction du gaspillage alimentaire et à l'adoption de comportements alimentaires plus sains.

\textbf{Objectifs du Projet}

Ce projet de fin d'année vise à développer une application mobile complète et intelligente, Hunger-Talk, qui répond aux besoins de gestion alimentaire et nutritionnelle. Les principaux objectifs de ce projet sont les suivants :

\begin{enumerate}
\item \textbf{Gestion Intelligente du Stock} : Créer un système permettant aux utilisateurs de suivre leur stock de produits alimentaires, avec des alertes pour les dates de péremption et des suggestions pour optimiser l'utilisation des produits disponibles.
\item \textbf{Recommandations Personnalisées} : Développer un système de recommandation de recettes basé sur le stock disponible, les préférences de l'utilisateur et ses objectifs nutritionnels, en exploitant une architecture RAG pour une personnalisation avancée.
\item \textbf{Assistant Conversationnel Intelligent} : Implémenter un assistant conversationnel basé sur un modèle LLaMA 3.1, capable de comprendre les requêtes naturelles des utilisateurs et de fournir des réponses contextuelles et pertinentes.
\item \textbf{Suivi Nutritionnel} : Mettre en place un système de suivi des objectifs nutritionnels, permettant aux utilisateurs de suivre leur consommation et d'ajuster leurs habitudes alimentaires en fonction de leurs objectifs.
\end{enumerate}

\textbf{Approche Méthodologique}

Pour atteindre ces objectifs, l'approche méthodologique de ce projet comprend plusieurs étapes clés :

\begin{enumerate}
\item \textbf{Analyse des Besoins} : Étudier les besoins réels des utilisateurs en matière de gestion alimentaire et identifier les fonctionnalités prioritaires à développer.
\item \textbf{Conception et Architecture} : Développer une architecture robuste utilisant Flutter pour l'application mobile, FastAPI pour le backend et PostgreSQL pour la base de données, intégrant un système RAG avec LLaMA 3.1.
\item \textbf{Développement} : Implémenter les fonctionnalités clés de l'application, en mettant l'accent sur l'expérience utilisateur, la personnalisation et l'intégration de l'intelligence artificielle.
\item \textbf{Tests et Validation} : Conduire des tests approfondis pour garantir la fiabilité, les performances et la pertinence des recommandations du système.
\item \textbf{Déploiement} : Déployer l'application et préparer sa mise à disposition pour les utilisateurs finaux.
\end{enumerate}

\textbf{Impact Attendu}

Le développement de cette application devrait avoir plusieurs impacts positifs :

\begin{itemize}
\item \textbf{Réduction du Gaspillage Alimentaire} : Optimisation de l'utilisation des produits disponibles et alertes pour les dates de péremption, contribuant à réduire les pertes alimentaires au niveau des ménages.
\item \textbf{Amélioration de l'Alimentation} : Accès facilité à des recettes adaptées et suivi des objectifs nutritionnels, encourageant l'adoption de comportements alimentaires plus sains.
\item \textbf{Expérience Utilisateur Enrichie} : Interface intuitive et assistant conversationnel intelligent, rendant la gestion alimentaire plus agréable et moins contraignante.
\item \textbf{Innovation Technologique} : Intégration de technologies d'intelligence artificielle avancées (RAG, LLM) dans une application mobile, démontrant le potentiel de ces technologies pour améliorer le quotidien des utilisateurs.
\end{itemize}

En conclusion, cette introduction générale prépare le terrain pour une étude détaillée qui explorera les technologies et les pratiques actuelles dans le domaine de la gestion alimentaire intelligente, de la recommandation de recettes et de l'intégration de l'intelligence artificielle dans les applications mobiles. En combinant cette étude avec l'expertise technique acquise au cours de ce projet et les besoins réels des utilisateurs, nous aspirons à créer une solution innovante qui contribuera significativement à l'amélioration de la gestion alimentaire et nutritionnelle au quotidien.
