\chapter*{Remerciements}
\markboth{Remerciements}{Remerciements}

Avant de commencer la présentation de ce modeste travail, je souhaite exprimer ma reconnaissance à toutes les personnes qui ont contribué, de près ou de loin, à la réalisation de ce projet de fin d'année consacré au développement de l'application Hunger-Talk.

Je tiens tout d'abord à adresser mes remerciements les plus sincères à mon encadrante de projet, Mme Benaddi, pour son accompagnement constant, ses conseils précieux et sa disponibilité tout au long de cette année. Sa rigueur, sa bienveillance et son exigence académique ont été des repères essentiels qui m'ont permis de structurer ce travail, de progresser méthodiquement et de surmonter les difficultés rencontrées, tant sur le plan technique que méthodologique.

Je souhaite également remercier chaleureusement l'équipe pédagogique de l'EMSI pour la qualité de l'enseignement dispensé et pour le cadre de travail qu'elle offre aux étudiants. Les connaissances théoriques et pratiques acquises au fil des années ont constitué une base solide pour la conception et la mise en œuvre de ce projet, en particulier dans les domaines du développement web, du mobile, des bases de données et de l'intelligence artificielle.

J'exprime aussi ma gratitude envers l'ensemble des enseignants et intervenants qui ont, par leurs cours, leurs projets et leurs retours, contribué à renforcer mes compétences et à affiner mon intérêt pour les thématiques liées à la nutrition, aux systèmes d'information et aux applications intelligentes. Leur engagement et leur investissement auprès des étudiants ont joué un rôle important dans la réussite de ce travail.

Enfin, je souhaite remercier ma famille, mes amis et mes camarades de promotion pour leur soutien, leurs encouragements et leurs conseils tout au long de cette année. Leur présence, leurs échanges et leur aide, qu'elle soit morale, technique ou simplement amicale, ont été d'un grand réconfort et ont largement contribué à mener à bien ce projet Hunger-Talk.
