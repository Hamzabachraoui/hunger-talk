\vspace*{\stretch{1}}
\begin{center}
	\begin{abstract}
	
        Ce projet de fin d'année, intitulé Hunger-Talk, consiste en le développement d'une application mobile intelligente dédiée à la gestion nutritionnelle et alimentaire. L'application permet aux utilisateurs de gérer leur stock de produits, de suivre leurs objectifs nutritionnels et de réduire le gaspillage alimentaire grâce à une interface moderne et accessible. Le système s'appuie sur un backend développé avec FastAPI et une base de données PostgreSQL, tandis que la partie mobile est réalisée en Flutter. Hunger-Talk intègre un assistant conversationnel basé sur un modèle LLaMA 3.1 et une architecture RAG (Retrieval Augmented Generation), qui exploite le stock, les préférences et les objectifs de l'utilisateur pour proposer des recettes pertinentes et des recommandations personnalisées. En combinant gestion de stock, recommandations culinaires, suivi nutritionnel et interaction avec l'IA, ce projet illustre la mise en place d'une solution complète et évolutive au service d'une alimentation plus saine et mieux organisée.
        
        \rule{\linewidth}{0.2 mm} \\[0.4 cm]
       
        \begin{center}\textbf{Abstract}\end{center} 
        
        This end-of-year project, named Hunger-Talk, focuses on the development of an intelligent mobile application dedicated to nutritional and food management. The application enables users to manage their food inventory, track their nutritional goals, and reduce food waste through a modern and user-friendly interface. The system is built on a FastAPI backend connected to a PostgreSQL database, while the mobile application is implemented in Flutter. Hunger-Talk integrates a conversational assistant powered by a LLaMA 3.1 model and a Retrieval Augmented Generation (RAG) architecture, leveraging the user's stock, preferences, and goals to provide relevant recipes and personalized recommendations. By combining stock management, culinary recommendations, nutritional monitoring, and AI-based interaction, this project demonstrates a complete and scalable solution that supports healthier and more organized eating habits.
        
        \rule{\linewidth}{0.2 mm} \\[0.4 cm]
        
        \textbf{Mots-clés :} Gestion nutritionnelle, Gestion de stock alimentaire, Réduction du gaspillage alimentaire, Recommandation de recettes, Intelligence artificielle, RAG (Retrieval Augmented Generation), LLaMA 3.1, Application mobile, Flutter, FastAPI, PostgreSQL, Assistant conversationnel, Suivi nutritionnel
        
        \rule{\linewidth}{0.2 mm} \\[0.4 cm]
        
        \textbf{Keywords :} Nutritional management, Food inventory management, Food waste reduction, Recipe recommendation, Artificial intelligence, RAG (Retrieval Augmented Generation), LLaMA 3.1, Mobile application, Flutter, FastAPI, PostgreSQL, Conversational assistant, Nutritional tracking
    \end{abstract}
\end{center}
\vspace*{\stretch{1}}
